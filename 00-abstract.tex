\begin{abstract}
Cardiac activation mapping is critical for guiding catheter ablation of arrhythmias but remains challenging due to the difficulty of reconstructing wavefront propagation from sparse intracardiac catheter scans. Existing methods often neglect the physical laws governing electrical signal propagation in cardiac tissues. Approaches that do incorporate physics rely on isotropic models, assuming constant conduction velocity in all directions, and thus fail to account for the influence of fiber orientation, leading to physiologically inconsistent activation maps. In this article, we propose a novel Physics-Informed Neural Network (PINN) framework for cardiac activation mapping that directly incorporates the anisotropic Eikonal equation into the learning process. This formulation embeds the underlying physics of cardiac conduction and explicitly models the effect of myocardial fiber orientation on wavefront propagation. The framework employs a dual-network architecture: one network estimates activation time $\phi(\mathbf{x})$, and the other reconstructs the direction-dependent conduction velocity tensor $\mathbf{D}(\mathbf{x})$ from sparse intracardiac catheter data. Evaluated on a synthetic 2D dataset with sharp conductivity transitions, the framework accurately reconstructed activation times (RMSE: 0.0083 ms) and captured key features of conduction anisotropy. While reconstruction of the full tensor field remains an area for improvement, this physics-consistent approach represents a significant step toward clinically reliable cardiac activation maps for ablation planning.
\end{abstract}