\section{Previous Work}

The reconstruction of cardiac activation maps from sparse intracardiac catheter data has evolved from simple interpolation methods to more advanced machine learning (ML) techniques, and most recently to physics-informed models. Early efforts used interpolation strategies such as radial basis functions (RBFs) and cosine-fit methods \cite{Mase2010} to estimate activation and conduction velocity maps. While computationally efficient, these methods failed to incorporate the underlying physics of electrophysiological wave propagation, making them unreliable for capturing nontrivial patterns such as curved or broken wavefronts.

To enhance robustness and handle uncertainty, later approaches adopted statistical models. Coveney et al. \cite{coveney2020} introduced a Gaussian Process-based probabilistic interpolation method that provided activation time estimates along with uncertainty quantification—an advantage in low-resolution or noisy scenarios. However, like geometric methods, these statistical models lacked any enforcement of physical laws, potentially leading to physiologically implausible predictions that violate known constraints of cardiac conduction.

To overcome these issues, supervised ML models were developed to directly learn mappings from electroanatomical data. Architectures such as SATDNN-AT and DirectMap \cite{Karoui2021} improved predictive accuracy and robustness in sparse data conditions. Yet, they do not enforce biophysical consistency and can produce activation maps that are visually plausible but physically invalid. Without explicitly embedding conduction physics, such models risk generating activation patterns that contradict the known dynamics of wavefront propagation in cardiac tissue.

Physics-Informed Neural Networks (PINNs) offer a hybrid solution by integrating mechanistic modeling with data-driven learning \cite{raissi2019}. Sahli Costabal et al. \cite{SahliCostabal2020} applied this approach by incorporating the isotropic Eikonal equation into a PINN framework, enabling simultaneous inference of activation times and scalar conduction velocity fields. By embedding the governing physics into the learning process, their model achieved improved accuracy, stability, and sample efficiency. However, a key limitation remains: the assumption of isotropic conduction ignores the directional dependence introduced by cardiac fiber architecture. Such models treat conduction velocity as uniform in all directions, neglecting the anisotropy that plays a central role in physiological and pathological activation dynamics.

Building on these advances, our work represents the natural evolution of cardiac activation mapping by extending the PINN framework to incorporate anisotropic conduction. Specifically, we integrate the anisotropic Eikonal equation \cite{ColliFranzone1990} directly into the learning process, enabling the model to account for the direction-dependent effects of myocardial fiber orientation, an aspect overlooked in previous PINN-based approaches that assumed isotropic conduction.