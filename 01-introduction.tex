\section{Introduction}

Cardiac arrhythmias represent a major global health burden, affecting over 59 million people worldwide and contributing significantly to cardiovascular morbidity and mortality \cite{roth2020}. Among these, atrial fibrillation (AF) is the most prevalent sustained arrhythmia, impacting millions of people all over the world. For example, in the US alone, 2.7–6.1 million Americans suffer from atrial fibrillation, with projections suggesting this number could rise to 12.1 million by 2030 \cite{roth2020, Colilla2013}. Ventricular arrhythmias like tachycardia (VT) affect around 300,000 patients annually in the US and account for 50\% of cardiovascular deaths \cite{AlKhatib2018}. Catheter ablation has become a cornerstone therapy for these conditions, with more than 500,000 procedures performed globally each year to disrupt abnormal electrical pathways \cite{Calkins2017}. The efficacy of ablation critically depends on accurate cardiac activation mapping, where clinicians record local activation times (LATs) at discrete cardiac sites using catheter electrodes.

Cardiac activation maps integrate LATs into a spatiotemporal representation of electrical wavefront propagation, delineating the sequence of myocardial depolarization through isochronal contours (lines of equal activation time) \cite{DURRER1970}. Conduction velocity (CV), derived from the spatial gradient of LATs, quantifies the speed (m/s) and direction of wavefront movement \cite{ColliFranzone1998}. Critically, myocardial tissue exhibits electrical anisotropy, wherein CV is heavily influenced by the orientation of cardiac fibers relative to the wavefront direction \cite{Spach1981}. CV is typically 2–3 times faster when propagation occurs parallel to fiber orientation (longitudinal direction) compared to perpendicular (transverse direction) due to higher intercellular coupling via gap junctions along fiber axes \cite{Spach1981,KLBER2004}. Consequently, activation maps must account for fiber orientation to avoid misinterpretation of conduction properties, as wavefronts traversing transverse to fibers may mimic pathological slowing or conduction block, even in healthy tissue \cite{Spach1981,Taccardi1994}.
