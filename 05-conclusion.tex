\section{Conclusion and Future Work}

In this article, we introduced an anisotropic physics-informed neural network (PINN) framework for reconstructing cardiac activation times and direction-dependent conduction velocity tensors from sparse intracardiac catheter data. By embedding the anisotropic Eikonal equation directly into the training process, our approach models the effects of myocardial fiber orientation, an important factor often neglected by existing isotropic PINN models.

We validated our method on a 2D experiment featuring sharp conductivity transitions. The model accurately reconstructed activation maps from sparse data (NRMSE: $1.76\%$) and captured essential features of conduction anisotropy, though the reconstruction of the conduction velocity tensor field remained limited.

Our results underscore the value of incorporating physiologically grounded models into neural network frameworks. This article provides a foundation for future work aimed at improving conduction velocity tensor field accuracy, and extending the approach to 3D cardiac geometries. Ultimately, our framework represents a promising step toward generating reliable, biophysically consistent activation maps to support catheter ablation planning and treatment of arrhythmias.